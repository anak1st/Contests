\documentclass[11pt, a4paper, oneside]{article}
\usepackage[UTF8]{ctex}
% xcolor宏包一般要放在最前面!否则那3张颜色表容易加不进来。
\usepackage[dvipsnames, svgnames, x11names]{xcolor} 
\usepackage{amsmath, amsthm, amssymb, graphicx}
\usepackage{listings}
\usepackage[bookmarks=true, colorlinks, citecolor=blue, linkcolor=black]{hyperref}

% 导言区
\title{我的第一个\LaTeX 文档}
\author{夏凡 Anak1st}
\date{\today}

\lstset{
    basicstyle          =   \sffamily,          % 基本代码风格
    keywordstyle        =   \bfseries,          % 关键字风格
    commentstyle        =   \rmfamily\itshape,  % 注释的风格,斜体
    stringstyle         =   \ttfamily,  % 字符串风格
    flexiblecolumns,                % 别问为什么,加上这个
    numbers             =   left,   % 行号的位置在左边
    showspaces          =   false,  % 是否显示空格,显示了有点乱,所以不现实了
    numberstyle         =   \zihao{-5}\ttfamily,    % 行号的样式,小五号,tt等宽字体
    showstringspaces    =   false,
    captionpos          =   t,      % 这段代码的名字所呈现的位置,t指的是top上面
    frame               =   lrtb,   % 显示边框
}

\lstdefinestyle{Python}{
    language        =   Python, % 语言选Python
    basicstyle      =   \zihao{-5}\ttfamily,
    numberstyle     =   \zihao{-5}\ttfamily,
    keywordstyle    =   \color{blue},
    keywordstyle    =   [2]\color{teal},
    stringstyle     =   \color{ForestGreen},
    commentstyle    =   \color{gray}\ttfamily,
    breaklines      =   true,   % 自动换行,建议不要写太长的行
    columns         =   fixed,  % 如果不加这一句,字间距就不固定,很丑,必须加
    basewidth       =   0.5em,
}

\begin{document}

\maketitle

\section{一级标题}

$ Y^{-1} A^{-1} D^{-1} \iff (DAY)^{-1} $

\subsection{用离散表示夺冠}
\begin{center}
    $P$ : 死, $Q$ : 夺冠, $F(x)$ : 是人 \\
    ${\forall}x(F(x) \wedge Q \Rightarrow P)$
\end{center}

\subsection{测试数学}
\begin{center}
    \begin{align}
        Y = & \bar{A} \bar{B} \bar{C} + A B C + \bar{B} C \bar{D} + \bar{A} B D \\
          = & \bar{A} \bar{B} \bar{C} \bar{D} + \bar{A} \bar{B} \bar{C} D \\
          + & A B C \bar{D} + A B C D \\
          + & \bar{A} \bar{B} C \bar{D} + A \bar{B} C \bar{D} \\
          + & \bar{A} B \bar{C} D + \bar{A} B C D \\
          = & m_0 + m_7 + \bar{D} (m_1 + m_5) + D (m_2 + m_3) \\
    \end{align}
\end{center}

\newpage
\subsection{代码}
\lstinputlisting[
    style       =   Python,
    caption     =   {\bf src/test/test.py},
    label       =   {src/test/test.py}
]{../../test/test.py}

\end{document}