\documentclass[11pt, a4paper, oneside]{article}
\usepackage[UTF8]{ctex}
% xcolor宏包一般要放在最前面!否则那3张颜色表容易加不进来。
\usepackage[dvipsnames, svgnames, x11names]{xcolor} 
\usepackage{amsmath, amsthm, amssymb, graphicx}
\usepackage{listings}
\usepackage[bookmarks=true, colorlinks, citecolor=blue, linkcolor=black]{hyperref}

% 导言区
\title{我的第一个\LaTeX 文档}
\author{夏凡 Anak1st}
\date{\today}

\lstset{
    basicstyle          =   \sffamily,          % 基本代码风格
    keywordstyle        =   \bfseries,          % 关键字风格
    commentstyle        =   \rmfamily\itshape,  % 注释的风格,斜体
    stringstyle         =   \ttfamily,  % 字符串风格
    flexiblecolumns,                % 别问为什么,加上这个
    numbers             =   left,   % 行号的位置在左边
    showspaces          =   false,  % 是否显示空格,显示了有点乱,所以不现实了
    numberstyle         =   \zihao{-5}\ttfamily,    % 行号的样式,小五号,tt等宽字体
    showstringspaces    =   false,
    captionpos          =   t,      % 这段代码的名字所呈现的位置,t指的是top上面
    frame               =   lrtb,   % 显示边框
}

\lstdefinestyle{Python}{
    language        =   Python, % 语言选Python
    basicstyle      =   \zihao{-5}\ttfamily,
    numberstyle     =   \zihao{-5}\ttfamily,
    keywordstyle    =   \color{blue},
    keywordstyle    =   [2]\color{teal},
    stringstyle     =   \color{ForestGreen},
    commentstyle    =   \color{gray}\ttfamily,
    breaklines      =   true,   % 自动换行,建议不要写太长的行
    columns         =   fixed,  % 如果不加这一句,字间距就不固定,很丑,必须加
    basewidth       =   0.5em,
}

\begin{document}

\maketitle

\section{一级标题}

$ Y^{-1} A^{-1} D^{-1} \iff (DAY)^{-1} $

\subsection{用离散表示夺冠}
\begin{center}
    $P$ : 死, $Q$ : 夺冠, $F(x)$ : 是人 \\
    ${\forall}x(F(x) \wedge Q \Rightarrow P)$
\end{center}

\subsection{测试数学}
\begin{center}
    $P(x)=\sum_{min}^{max}f(x)$ \\
    $P(x)=\int_{min}^{max}f(x)$ \\
    $(v(L_1, L_1 + w - 1) \times v(l_i, r_i) + v(L_2, L_2 + w - 1)) \equiv k \mod 9$
\end{center}

\newpage
\subsection{代码}
\lstinputlisting[
    style       =   Python,
    caption     =   {\bf src/test/test.py},
    label       =   {src/test/test.py}
]{../../test/test.py}

\section{数据库}

\subsection{关系代数}

\begin{enumerate}
\item $ S $
\item $ \Pi_{SN}{(\sigma_{AGE \geq 20}{(S)})} $
\item $ \Pi_{CNO}{(\sigma_{CPNO = 'C2'}{(C)})} $
\item $ \Pi_{SN}{(\sigma_{CNO = 'C1' \land SCORE = 'A'}{(SC)} \infty S)} $
\item $ \Pi_{SN,CPNO}{(\sigma_{SNO='S1'}{(SC)} \infty C)} $
\item $ \Pi_{CN}{(\sigma_{AGE = 23}{(S)} \infty SC \infty C)} $
\item $ \Pi_{SN}{(\Pi_{CNO}{(\sigma_{SNO = 'S5'}{(SC)})} \infty SC \infty S)} $
\item $ \Pi_{SN}{((SC \div \Pi_{CNO}{(\sigma_{SNO = 'S4'}{(SC)})}) \infty S)} $
\item $ \Pi_{SNO}{(SC \div \Pi_{CNO}{(C)})} $
\item $ \Pi_{SNO}{(S)} - \Pi_{SNO}{(SC)} $
\end{enumerate}

\subsection{关系演算}

\begin{enumerate}
	\item GET W(S)
	\item GET W(S.SN): S.AGE $\geq$ 20 
	\item GET W(C.CNO): C.CPNO=`C2'
	\item RANGE SC SCX \\
	GET W(S.SN): $\exists$ SCX(SCX.SNO=S.SNO $\wedge$ SCX.CNO=`C1' $\wedge$ SCX.SCORE=`A')
	\item RANGE SC SCX \\
	GET W(C.CN,C.CPNO): $\exists$ SCX(C.CNO=SCX.CNO $\wedge$ SCX.SNO=`S1')
	\item RANGE S SX \\
	\hspace*{1.4cm} SC SCX \\
	GET W(C.CN): $\exists$ SX $\exists$ SCX (C.CNO=SCX.CNO $\wedge$ SX.SNO=SCX.SNO
    $\wedge$ SX.AGE=23)
    
    \item RANGE SC SCX \\
    \hspace*{1.4cm} SC SCY \\
    GET W(S.SN): $\exists$ SCX $\exists$ SCY (SCX.SNO='S5' $\wedge$ S.SNO=SCY.SNO $\wedge$ SCY.CNO=SCX.CNO)
	
	\item RANGE C CX \\
	\hspace*{1.4cm} SC SCX \\
	\hspace*{1.4cm} SC SCY \\
	GET W(S.SN): $\forall$ CX($\exists$ SCX(SCX.CNO=CX.CNO $\wedge$ SCX.SNO='S4') $\rightarrow$ $\exists$ SCY(SCY.CNO=CX.CNO $\exists$ SCY.SNO=S.SNO))
	
	
\end{enumerate}

\end{document}